\documentclass[slides]{beamer}\usepackage[]{graphicx}\usepackage[]{color}

%\usepackage{pgfpages}
%\pgfpagesuselayout{2 on 1}[border shrink=5mm]
%\pgfpageslogicalpageoptions{1}{border code=\pgfusepath{stroke}}
%\pgfpageslogicalpageoptions{2}{border code=\pgfusepath{stroke}}

%% maxwidth is the original width if it is less than linewidth
%% otherwise use linewidth (to make sure the graphics do not exceed the margin)

\makeatletter
\def\maxwidth{ %
  \ifdim\Gin@nat@width>\linewidth
    \linewidth
  \else
    \Gin@nat@width
  \fi
}
\makeatother

\definecolor{fgcolor}{rgb}{0.345, 0.345, 0.345}
\newcommand{\hlnum}[1]{\textcolor[rgb]{0.686,0.059,0.569}{#1}}%
\newcommand{\hlstr}[1]{\textcolor[rgb]{0.192,0.494,0.8}{#1}}%
\newcommand{\hlcom}[1]{\textcolor[rgb]{0.678,0.584,0.686}{\textit{#1}}}%
\newcommand{\hlopt}[1]{\textcolor[rgb]{0,0,0}{#1}}%
\newcommand{\hlstd}[1]{\textcolor[rgb]{0.345,0.345,0.345}{#1}}%
\newcommand{\hlkwa}[1]{\textcolor[rgb]{0.161,0.373,0.58}{\textbf{#1}}}%
\newcommand{\hlkwb}[1]{\textcolor[rgb]{0.69,0.353,0.396}{#1}}%
\newcommand{\hlkwc}[1]{\textcolor[rgb]{0.333,0.667,0.333}{#1}}%
\newcommand{\hlkwd}[1]{\textcolor[rgb]{0.737,0.353,0.396}{\textbf{#1}}}%

\usepackage{framed}
\makeatletter
\newenvironment{kframe}{%
 \def\at@end@of@kframe{}%
 \ifinner\ifhmode%
  \def\at@end@of@kframe{\end{minipage}}%
  \begin{minipage}{\columnwidth}%
 \fi\fi%
 \def\FrameCommand##1{\hskip\@totalleftmargin \hskip-\fboxsep
 \colorbox{shadecolor}{##1}\hskip-\fboxsep
     % There is no \\@totalrightmargin, so:
     \hskip-\linewidth \hskip-\@totalleftmargin \hskip\columnwidth}%
 \MakeFramed {\advance\hsize-\width
   \@totalleftmargin\z@ \linewidth\hsize
   \@setminipage}}%
 {\par\unskip\endMakeFramed%
 \at@end@of@kframe}
\makeatother

\definecolor{shadecolor}{rgb}{.97, .97, .97}
\definecolor{messagecolor}{rgb}{0, 0, 0}
\definecolor{warningcolor}{rgb}{1, 0, 1}
\definecolor{errorcolor}{rgb}{1, 0, 0}
\newenvironment{knitrout}{}{} % an empty environment to be redefined in TeX

\usepackage{alltt}

\mode<presentation>
{
  %\usetheme[secheader]{Boadilla}
	%\usecolortheme[rgb={.835, .102,.169}]{structure}  
	\usetheme[width= 0cm]{Goettingen}
	%\setbeamercovered{transparent}
}
\setbeamertemplate{navigation symbols}{}
\setbeamertemplate{footline}[frame number]

\definecolor{blue2}{rgb}{0.278,0.278,0.729} 
\newcommand{\blue}[1]{\textcolor{blue2}{#1}}

\title{Lecture 1.1: Laying the Foundations + Terminology}
\author{Chapters 1.1-1.2}
\date{2014/01/27}
\IfFileExists{upquote.sty}{\usepackage{upquote}}{}


\begin{document}
%------------------------------------------------------------------------------
\begin{frame}
\titlepage
\end{frame}
%------------------------------------------------------------------------------


%------------------------------------------------------------------------------
\begin{frame}
\frametitle{Goals for Today}
\begin{itemize}
  \item Go over the syllabus 
  \pause\item Show some fun examples 
  \pause\item Discuss how to evaluate the efficacy of a \blue{treatment}
  \pause\item Describe the different kinds of \blue{variables} we'll consider
\end{itemize}

\end{frame}
%------------------------------------------------------------------------------


%------------------------------------------------------------------------------
\begin{frame}
\frametitle{What is statistics?}

(Direct from text) The general scientific process of investigation can be summed up as follows:

\begin{enumerate}
\pause\item Identify the scientific question or problem
\pause\item Collect relevant data on the topic
\pause\item Analyze the data
\pause\item Form a conclusion \pause and \blue{communicate it}
\end{enumerate}

\pause Statistics concerns itself with points 2 through 4.

\end{frame}
%------------------------------------------------------------------------------


%------------------------------------------------------------------------------
\begin{frame}[fragile]
\frametitle{Examples}
As of 2014/01/26, here are your majors.  I'll try to pick relevant examples throughout the course:

\begin{small}
\begin{verbatim}
                      Biology Biochem and Molecular Biology 
                           12                             6 
                    Economics                   Mathematics 
                            5                             4 
                   Psychology         Environmental Studies 
                            4                             3 
 International Policy Studies                     Sociology 
                            2                             2 
   Environmental Studies-Biol                         Music 
                            1                             1 
            Political Science                      Religion 
                            1                             1 
                    Chemistry    Environmental Studies-Econ 
                            1                             1 
                      Physics 
                            1 
\end{verbatim}
\end{small}
\end{frame}
%------------------------------------------------------------------------------


%------------------------------------------------------------------------------
\begin{frame}[fragile]
\frametitle{Example: Brain Cancer in Western Washington}
\begin{center}
\includegraphics[width=9cm]{figure/brain_post_high}
\end{center}
\end{frame}
%------------------------------------------------------------------------------


%------------------------------------------------------------------------------
\begin{frame}[fragile]
\frametitle{Example: Breast Cancer in Western Washington}
\begin{center}
\includegraphics[width=9cm]{figure/breast_post_high}
\end{center}
\end{frame}
%------------------------------------------------------------------------------


%------------------------------------------------------------------------------
\begin{frame}[fragile]
\frametitle{Example: 2012 Election - Nate Silver's Predictions vs Actual Results}
\begin{center}
\includegraphics[width=\textwidth]{figure/nate_silver.jpg}
\end{center}
\begin{center}
\end{center}
\end{frame}
%------------------------------------------------------------------------------


%------------------------------------------------------------------------------
\begin{frame}[fragile]
\frametitle{Example: Social Network Display of a Recent Party I Had}
\begin{center}
\includegraphics[width=8cm]{figure/network.png}
\end{center}
\end{frame}
%------------------------------------------------------------------------------


%------------------------------------------------------------------------------
\begin{frame}
\frametitle{Say we want answer the following questions:}
\begin{itemize}
\pause\item Will reassuring potential new users to a gambling website that we won't spam them increase the sign-up rate?
\pause\item Does a new kind of cognitive therapy alter levels of depression in patients?
\pause\item Or you question the effectiveness of ...
\end{itemize}

\begin{center}
\includegraphics[height=4cm]{figure/antioxidants.png}\includegraphics[height=4cm]{figure/atkins.jpg}
\end{center}

\end{frame}
%------------------------------------------------------------------------------



%------------------------------------------------------------------------------
\begin{frame}
\frametitle{Evaluating the efficacy of a `treatment'}
In all the above cases, you are questioning the efficacy of a \blue{treatment/intervention}.  One way to evaluate the efficacy is via an \blue{experiment} where you define
\begin{itemize}
\pause\item A \blue{control} group: the ``business as usual'' baseline group
\pause\item A \blue{treatment} group:  the group that receives/is subject to the treatment/intervention
\end{itemize}
\pause and make comparisons.

\end{frame}
%------------------------------------------------------------------------------



%------------------------------------------------------------------------------
\begin{frame}
\frametitle{Example of a treatment vs control}
\begin{center}
\includegraphics[width=9cm]{figure/control_treatment.png}
\end{center}
\end{frame}
%------------------------------------------------------------------------------



%------------------------------------------------------------------------------
\begin{frame}
\frametitle{Variables}
A \blue{variable} is a description of any characteristic whose value may change from one unit in the population to the next:
\begin{enumerate}
\pause\item gender of an engineer:  \blue{categorical} variable
\pause\item level of education (high school/GED, college, grad school): \blue{ordinal} variable
\pause\item number of major defects on a newly manufactured phone: \blue{discrete} variable i.e. something you can count
\pause\item temperature of the battery in a phone after 1 hour of use: \blue{continuous} variable; its possible values consist of an interval on the number line
\end{enumerate}

\end{frame}
%------------------------------------------------------------------------------



%------------------------------------------------------------------------------
\begin{frame}[fragile]
\frametitle{Variables Flow Chart}
\begin{knitrout}
\definecolor{shadecolor}{rgb}{0.969, 0.969, 0.969}\color{fgcolor}
\includegraphics[width=\linewidth]{figure/flow-chart} 
\end{knitrout}

\end{frame}
%------------------------------------------------------------------------------



%%------------------------------------------------------------------------------
%\begin{frame}[fragile]
%\frametitle{Practical}
%It is important to have good descriptions of the variables.  
%
%For example in the \verb#county# data set, we have \verb#income# \verb#med_income#.  What do these mean?  
%
%\begin{itemize}
%  \item \verb#income#: income per capita
%  \item \verb#med_income#: median household income for the county, where a household's income equals the total income of its occupants who are 15 years or older.  
%\end{itemize}
%
%\end{frame}
%%------------------------------------------------------------------------------



%------------------------------------------------------------------------------
\begin{frame}
\frametitle{Data}

At its simplest, data are presented in a data table or matrix where (almost always) each
\begin{itemize}
\pause\item row corresponds to \blue{cases} or \blue{units of observation/analysis}
\pause\item column represents the variables corresponding to a particular observation
\end{itemize}

\pause It is almost always the case that 

\begin{itemize}
  \pause\item $n$ is the number of observations
  \pause\item $p$ is the number of variables
\end{itemize}

\end{frame}
%------------------------------------------------------------------------------




%------------------------------------------------------------------------------
\begin{frame}[fragile]
\frametitle{Data Summaries}

Consider the variable "federal spending per capita" in each of the 3,143 counties in the US.  One can hardly digest this:

\begin{tiny}
\begin{verbatim}
   [1]   6.068095   6.139862   8.752158   7.122016   5.130910   9.973062   9.311835  15.439218
   [9]   8.613707   7.104621   6.324061  10.640378   9.781442   8.982702   6.840035  20.330684
  [17]   9.687698  11.080738   7.839761   9.461856   9.650295   7.760627  25.774791  13.948106
   ...
[3121]   7.520731  10.246400   3.106800  17.679572   4.824044   7.247212   8.484211   8.794626
[3129]   9.829593   8.100945  17.090715   4.855849   6.621378  22.587359  10.813260  11.422522
[3137]   9.580265   4.368986   5.062138   6.236968   4.549105   8.713817   6.694784
\end{verbatim}
\end{tiny}

\end{frame}
%------------------------------------------------------------------------------




%------------------------------------------------------------------------------
\begin{frame}[fragile]
\frametitle{Data Summaries}
We can't interpret all the data at once; we need to boil it down via \blue{summary statistics}, single numbers summarizing a large amount of data.

\vspace{0.5cm}

\pause Using the \verb#summary()# command in \verb#R#:

\begin{small}
\begin{verbatim}
   Min. 1st Qu.  Median    Mean 3rd Qu.    Max.    NA's 
  0.000   6.964   8.669   9.991  10.860 204.600       4 
\end{verbatim}
\end{small}


\end{frame}
%------------------------------------------------------------------------------


%------------------------------------------------------------------------------
\begin{frame}[fragile]
\frametitle{Relationships between variables}
We can best display the relationship between two variables using a \blue{scatterplot AKA bivariate plot}:

\begin{center}
\pause\includegraphics[width=\linewidth]{figure/relationships} 
\end{center}


\end{frame}
%------------------------------------------------------------------------------


%------------------------------------------------------------------------------
\begin{frame}
\frametitle{Relationships between variables}
Almost always we are interested in the relationship between two or more variables.

\vspace{0.25cm}

\pause A pair of variables are either related in some way (\blue{associated}) or not (\blue{independent}).  No pair of variables are both associated and independent.   

\vspace{0.25cm}

\pause We can have either a \blue{negative association} (as the value of one variable increases, the other decreases) or a \blue{positive association}.

\end{frame}
%------------------------------------------------------------------------------


%------------------------------------------------------------------------------
\begin{frame}[fragile]
\frametitle{Relationships between variables}
We can consider a third variable in the previous plot.
\begin{center}
\includegraphics[width=\textwidth]{figure/MHP.png}
\end{center}
\end{frame}
%------------------------------------------------------------------------------


%------------------------------------------------------------------------------
\begin{frame}
\frametitle{Next Time}
We will build on today's terminology to

\begin{itemize}
  \pause\item Understand important considerations about data collection
  \pause\item In particular we will discuss sampling.
\end{itemize}

\end{frame}
%------------------------------------------------------------------------------


\end{document}









