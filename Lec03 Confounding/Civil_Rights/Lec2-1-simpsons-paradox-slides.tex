\documentclass[slides]{beamer}\usepackage[]{graphicx}\usepackage[]{color}

%\usepackage{pgfpages}
%\pgfpagesuselayout{2 on 1}[border shrink=5mm]
%\pgfpageslogicalpageoptions{1}{border code=\pgfusepath{stroke}}
%\pgfpageslogicalpageoptions{2}{border code=\pgfusepath{stroke}}

%% maxwidth is the original width if it is less than linewidth
%% otherwise use linewidth (to make sure the graphics do not exceed the margin)
\makeatletter
\def\maxwidth{ %
  \ifdim\Gin@nat@width>\linewidth
    \linewidth
  \else
    \Gin@nat@width
  \fi
}
\makeatother

\definecolor{fgcolor}{rgb}{0.345, 0.345, 0.345}
\newcommand{\hlnum}[1]{\textcolor[rgb]{0.686,0.059,0.569}{#1}}%
\newcommand{\hlstr}[1]{\textcolor[rgb]{0.192,0.494,0.8}{#1}}%
\newcommand{\hlcom}[1]{\textcolor[rgb]{0.678,0.584,0.686}{\textit{#1}}}%
\newcommand{\hlopt}[1]{\textcolor[rgb]{0,0,0}{#1}}%
\newcommand{\hlstd}[1]{\textcolor[rgb]{0.345,0.345,0.345}{#1}}%
\newcommand{\hlkwa}[1]{\textcolor[rgb]{0.161,0.373,0.58}{\textbf{#1}}}%
\newcommand{\hlkwb}[1]{\textcolor[rgb]{0.69,0.353,0.396}{#1}}%
\newcommand{\hlkwc}[1]{\textcolor[rgb]{0.333,0.667,0.333}{#1}}%
\newcommand{\hlkwd}[1]{\textcolor[rgb]{0.737,0.353,0.396}{\textbf{#1}}}%

\usepackage{framed}
\makeatletter
\newenvironment{kframe}{%
 \def\at@end@of@kframe{}%
 \ifinner\ifhmode%
  \def\at@end@of@kframe{\end{minipage}}%
  \begin{minipage}{\columnwidth}%
 \fi\fi%
 \def\FrameCommand##1{\hskip\@totalleftmargin \hskip-\fboxsep
 \colorbox{shadecolor}{##1}\hskip-\fboxsep
     % There is no \\@totalrightmargin, so:
     \hskip-\linewidth \hskip-\@totalleftmargin \hskip\columnwidth}%
 \MakeFramed {\advance\hsize-\width
   \@totalleftmargin\z@ \linewidth\hsize
   \@setminipage}}%
 {\par\unskip\endMakeFramed%
 \at@end@of@kframe}
\makeatother

\definecolor{shadecolor}{rgb}{.97, .97, .97}
\definecolor{messagecolor}{rgb}{0, 0, 0}
\definecolor{warningcolor}{rgb}{1, 0, 1}
\definecolor{errorcolor}{rgb}{1, 0, 0}
\newenvironment{knitrout}{}{} % an empty environment to be redefined in TeX

\usepackage{alltt}

\mode<presentation>
{
  %\usetheme[secheader]{Boadilla}
	%\usecolortheme[rgb={.835, .102,.169}]{structure}  
	\usetheme[width= 0cm]{Goettingen}
	%\setbeamercovered{transparent}
}
\setbeamertemplate{navigation symbols}{}

\definecolor{blue2}{rgb}{0.278,0.278,0.729} 
\newcommand{\blue}[1]{\textcolor{blue2}{#1}}

\usepackage{hyperref}




\begin{document}











%------------------------------------------------------------------------------
\begin{frame}
\frametitle{Civil Rights Act Vote Breakdown}

\begin{center}
	\begin{tabular}{c|ccc|ccc}
     & \multicolumn{3}{c|}{Democrats}  & \multicolumn{3}{c}{Republicans} \\ 
     & Pro & Total & \% & Pro & Total & \% \\ 
     \hline
     \includegraphics[height=0.4cm]{conf} & 8 & 91 & 8\% & 0 & 11 & 0\% \\ 
	 \includegraphics[height=0.4cm]{union} & 144 & 152 & 95\% & 137 & 161 & 85\% \\ 
    \hline
     Total & 152 & 243 & 62\% & 137 & 172 & 80\% \\ 
  \end{tabular}
\end{center}

\textcolor{white}{We observe that} 
\begin{enumerate}
\item[] \textcolor{white}{91/(91 + 152) = 37\% of Democrats}
\item[] \textcolor{white}{11/(11+161) = 6\% of Republicans}
\end{enumerate}
\textcolor{white}{were from former confederate states.}

\end{frame}
%------------------------------------------------------------------------------




%------------------------------------------------------------------------------
\begin{frame}
\frametitle{Civil Rights Act Vote Breakdown}

\begin{center}
	\begin{tabular}{c|ccc|ccc}
     & \multicolumn{3}{c|}{Democrats}  & \multicolumn{3}{c}{Republicans} \\ 
     & Pro & Total & \% & Pro & Total & \% \\ 
     \hline
     \includegraphics[height=0.4cm]{conf} & 8 & 91 & \blue{8\%} & 0 & 11 & 0\% \\ 
	 \includegraphics[height=0.4cm]{union} & 144 & 152 & \blue{95\%} & 137 & 161 & 85\% \\ 
    \hline
     Total & 152 & 243 & 62\% & 137 & 172 & \blue{80\%} \\ 
  \end{tabular}
\end{center}

\textcolor{white}{We observe that} 
\begin{enumerate}
\item[] \textcolor{white}{91/(91 + 152) = 37\% of Democrats}
\item[] \textcolor{white}{11/(11+161) = 6\% of Republicans}
\end{enumerate}
\textcolor{white}{were from former confederate states.}

\end{frame}
%------------------------------------------------------------------------------





%------------------------------------------------------------------------------
\begin{frame}
\frametitle{Civil Rights Act Vote Breakdown}

\begin{center}
	\begin{tabular}{c|ccc|ccc}
     & \multicolumn{3}{c|}{Democrats}  & \multicolumn{3}{c}{Republicans} \\ 
     & Pro & Total & \% & Pro & Total & \% \\ 
     \hline
     \includegraphics[height=0.4cm]{conf} & 8 & \textcolor{red}{91} & \blue{8\%} & 0 & \textcolor{red}{11} & 0\% \\ 
	 \includegraphics[height=0.4cm]{union} & 144 & \textcolor{red}{152} & \blue{95\%} & 137 & \textcolor{red}{161} & 85\% \\ 
    \hline
     Total & 152 & 243 & 62\% & 137 & 172 & \blue{80\%} \\ 
  \end{tabular}
\end{center}

We observe that 
\begin{enumerate}
\item \textcolor{red}{91/(91 + 152) = 37\%} of Democrats
\item \textcolor{red}{11/(11+161) = 6\%} of Republicans
\end{enumerate}
were from former confederate states.

\end{frame}
%------------------------------------------------------------------------------



%------------------------------------------------------------------------------
\begin{frame}
\frametitle{Confederate vs Union States}

This is not a fair comparison between Dems and Repubs!  Why?\pause

\vspace{0.25cm}

\begin{center}
	\begin{tabular}{c|ccc|ccc}
     & \multicolumn{3}{c|}{\includegraphics[height=0.4cm]{conf}}  & \multicolumn{3}{c}{\includegraphics[height=0.4cm]{union}} \\ 
     & Pro & Total & \% & Pro & Total & \% \\ 
    \hline
     Total & 8 & 102 & 8\% & 281 & 313 & \blue{90\%} \\ 
     \multicolumn{7}{c}{}
  \end{tabular}
\end{center}

\vspace{0.1cm}

There is an enormous difference in support for the Civil Rights Act between Confederate and Union states!

\end{frame}
%------------------------------------------------------------------------------



%------------------------------------------------------------------------------
\begin{frame}
\frametitle{Not a Fair Comparison}
So...

\begin{center}
	\begin{tabular}{c|ccc|ccc}
     & \multicolumn{3}{c|}{Democrats}  & \multicolumn{3}{c}{Republicans} \\ 
     & Pro & Total & \% & Pro & Total & \% \\ 
     \hline
     \includegraphics[height=0.4cm]{conf} & 8 & \textcolor{red}{91} & \blue{8\%} & 0 & \textcolor{red}{11} & 0\% \\ 
	 \includegraphics[height=0.4cm]{union} & 144 & \textcolor{red}{152} & \blue{95\%} & 137 & \textcolor{red}{161} & 85\% \\ 
    \hline
     Total & 152 & 243 & 62\% & 137 & 172 & \blue{80\%} \\ 
  \end{tabular}
\end{center}
is not an apples to apples comparison!  We need to level the playing field.

\end{frame}
%------------------------------------------------------------------------------


%------------------------------------------------------------------------------
\begin{frame}
\frametitle{Leveling the Playing Field}
\begin{small}
So let's imagine a hypothetical world of Repubs where \pause
\begin{enumerate}
\item we have the same number of Repubs (172) \pause
\item the Repub voting percentages stay the same (0\% and 85\%) \pause
\item CRUCIAL: the Confederate/Union mix is the same as with the Dems (\textcolor{red}{91 vs 152} i.e. 37\% vs 63\%) \pause
\end{enumerate}
\end{small}


\begin{center}
	\begin{tabular}{c|ccc|ccc}
     & \multicolumn{3}{c|}{Democrats}  & \multicolumn{3}{c}{Republicans} \\ 
     & Pro & Total & \% & Pro & Total & \% \\ 
     \hline
     \includegraphics[height=0.4cm]{conf} & 8 & \textcolor{red}{91} & 8\% &  &  & 0\% \\ 
	 \includegraphics[height=0.4cm]{union} & 144 & \textcolor{red}{152} & 95\% &  &  & 85\% \\ 
    \hline
     Total & 152 & 243 & 62\% & & 172 & \blue{} \\ 
  \end{tabular}
\end{center}

\begin{enumerate}
\item[] \textcolor{white}{91/(91 + 152) = 37\% $\times$ 172 = 64}
\item[] \textcolor{white}{172 - 64 = 108}
\item[] \textcolor{white}{64 $\times$ 0\% = 0 and 108 $\times$ 85\% = 92}
\item[] \textcolor{white}{92/(92+172) = 53\%}
\end{enumerate}

\end{frame}
%------------------------------------------------------------------------------


%------------------------------------------------------------------------------
\begin{frame}
\frametitle{Leveling the Playing Field}
\begin{small}
So let's imagine a hypothetical world of Repubs where
\begin{enumerate}
\item we have the same number of Repubs (172)
\item the Repub voting percentages stay the same (0\% and 85\%)
\item CRUCIAL: the Confederate/Union mix is the same as with the Dems (\textcolor{red}{91 vs 152} i.e. 37\% vs 63\%) 
\end{enumerate}
\end{small}


\begin{center}
	\begin{tabular}{c|ccc|ccc}
     & \multicolumn{3}{c|}{Democrats}  & \multicolumn{3}{c}{Republicans} \\ 
     & Pro & Total & \% & Pro & Total & \% \\ 
     \hline
     \includegraphics[height=0.4cm]{conf} & 8 & \textcolor{red}{91} & 8\% &  & 64 & 0\% \\ 
	 \includegraphics[height=0.4cm]{union} & 144 & \textcolor{red}{152} & 95\% &  &  & 85\% \\ 
    \hline
     Total & 152 & 243 & 62\% & & 172 & \blue{} \\ 
  \end{tabular}
\end{center}

\begin{enumerate}
\item 37\% $\times$ 172 = 64
\item[] \textcolor{white}{172 - 64 = 108}
\item[] \textcolor{white}{64 $\times$ 0\% = 0 and 108 $\times$ 85\% = 92}
\item[] \textcolor{white}{92/(92+172) = 53\%}
\end{enumerate}

\end{frame}
%------------------------------------------------------------------------------


%------------------------------------------------------------------------------
\begin{frame}
\frametitle{Leveling the Playing Field}
\begin{small}
So let's imagine a hypothetical world of Repubs where
\begin{enumerate}
\item we have the same number of Repubs (172)
\item the Repub voting percentages stay the same (0\% and 85\%)
\item CRUCIAL: the Confederate/Union mix is the same as with the Dems (\textcolor{red}{91 vs 152} i.e. 37\% vs 63\%) 
\end{enumerate}
\end{small}


\begin{center}
	\begin{tabular}{c|ccc|ccc}
     & \multicolumn{3}{c|}{Democrats}  & \multicolumn{3}{c}{Republicans} \\ 
     & Pro & Total & \% & Pro & Total & \% \\ 
     \hline
     \includegraphics[height=0.4cm]{conf} & 8 & \textcolor{red}{91} & 8\% &  & 64 & 0\% \\ 
	 \includegraphics[height=0.4cm]{union} & 144 & \textcolor{red}{152} & 95\% &  & 108 & 85\% \\ 
    \hline
     Total & 152 & 243 & 62\% & & 172 & \blue{} \\ 
  \end{tabular}
\end{center}

\begin{enumerate}
\item 37\% $\times$ 172 = 64
\item 172 - 64 = 108
\item[] \textcolor{white}{64 $\times$ 0\% = 0 and 108 $\times$ 85\% = 92}
\item[] \textcolor{white}{92/(92+172) = 53\%}
\end{enumerate}

\end{frame}
%------------------------------------------------------------------------------


%------------------------------------------------------------------------------
\begin{frame}
\frametitle{Leveling the Playing Field}
\begin{small}
So let's imagine a hypothetical world of Repubs where
\begin{enumerate}
\item we have the same number of Repubs (172)
\item the Repub voting percentages stay the same (0\% and 85\%)
\item CRUCIAL: the Confederate/Union mix is the same as with the Dems (\textcolor{red}{91 vs 152} i.e. 37\% vs 63\%) 
\end{enumerate}
\end{small}


\begin{center}
	\begin{tabular}{c|ccc|ccc}
     & \multicolumn{3}{c|}{Democrats}  & \multicolumn{3}{c}{Republicans} \\ 
     & Pro & Total & \% & Pro & Total & \% \\ 
     \hline
     \includegraphics[height=0.4cm]{conf} & 8 & \textcolor{red}{91} & 8\% & 0 & 64 & 0\% \\ 
	 \includegraphics[height=0.4cm]{union} & 144 & \textcolor{red}{152} & 95\% & 92 & 108 & 85\% \\ 
    \hline
     Total & 152 & 243 & 62\% & 92 & 172 & \blue{} \\ 
  \end{tabular}
\end{center}

\begin{enumerate}
\item 37\% $\times$ 172 = 64
\item 172 - 64 = 108
\item 64 $\times$ 0\% = 0 and 108 $\times$ 85\% = 92
\item[] \textcolor{white}{92/(92+172) = 53\%}
\end{enumerate}

\end{frame}
%------------------------------------------------------------------------------


%------------------------------------------------------------------------------
\begin{frame}
\frametitle{Leveling the Playing Field}
\begin{small}
So let's imagine a hypothetical world of Repubs where
\begin{enumerate}
\item we have the same number of Repubs (172)
\item the Repub voting percentages stay the same (0\% and 85\%)
\item CRUCIAL: the Confederate/Union mix is the same as with the Dems (\textcolor{red}{91 vs 152} i.e. 37\% vs 63\%) 
\end{enumerate}
\end{small}


\begin{center}
	\begin{tabular}{c|ccc|ccc}
     & \multicolumn{3}{c|}{Democrats}  & \multicolumn{3}{c}{Republicans} \\ 
     & Pro & Total & \% & Pro & Total & \% \\ 
     \hline
     \includegraphics[height=0.4cm]{conf} & 8 & \textcolor{red}{91} & 8\% & 0 & 64 & 0\% \\ 
	 \includegraphics[height=0.4cm]{union} & 144 & \textcolor{red}{152} & 95\% & 92 & 108 & 85\% \\ 
    \hline
     Total & 152 & 243 & 62\% & 92 & 172 & 53\% \\ 
  \end{tabular}
\end{center}

\begin{enumerate}
\item 37\% $\times$ 172 = 64
\item 172 - 64 = 108
\item 64 $\times$ 0\% = 0 and 108 $\times$ 85\% = 92
\item 92/(92+172) = 53\%
\end{enumerate}

\end{frame}
%------------------------------------------------------------------------------


%------------------------------------------------------------------------------
\begin{frame}
\frametitle{Leveling the Playing Field}
\begin{small}
So let's imagine a hypothetical world of Repubs where
\begin{enumerate}
\item we have the same number of Repubs (172)
\item the Repub voting percentages stay the same (0\% and 85\%)
\item CRUCIAL: the Confederate/Union mix is the same as with the Dems (\textcolor{red}{91 vs 152} i.e. 37\% vs 63\%) 
\end{enumerate}
\end{small}


\begin{center}
	\begin{tabular}{c|ccc|ccc}
     & \multicolumn{3}{c|}{Democrats}  & \multicolumn{3}{c}{Republicans} \\ 
     & Pro & Total & \% & Pro & Total & \% \\ 
     \hline
     \includegraphics[height=0.4cm]{conf} & 8 & \textcolor{red}{91} & \blue{8\%} & 0 & 64 & 0\% \\ 
	 \includegraphics[height=0.4cm]{union} & 144 & \textcolor{red}{152} & \blue{95\%} & 92 & 108 & 85\% \\ 
    \hline
     Total & 152 & 243 & \blue{62\%} & 92 & 172 & 53\% \\ 
  \end{tabular}
\end{center}

\begin{enumerate}
\item 37\% $\times$ 172 = 64
\item 172 - 64 = 108
\item 64 $\times$ 0\% = 0 and 108 $\times$ 85\% = 92
\item 92/(92+172) = 53\%
\end{enumerate}

\end{frame}
%------------------------------------------------------------------------------




%------------------------------------------------------------------------------
\begin{frame}
\frametitle{Summary}

\blue{Real data}: different Confederate/Union proportions between Dems/Repubs
\begin{center}
	\begin{tabular}{c|ccc|ccc}
     & \multicolumn{3}{c|}{Democrats}  & \multicolumn{3}{c}{Republicans} \\ 
     & Pro & Total & \% & Pro & Total & \% \\ 
     \hline
     Confederate & 8 & 91 & \blue{8\%} & 0 & \textcolor{red}{11} & 0\% \\ 
	 Union & 144 & 152 & \blue{95\%} & 137 & \textcolor{red}{161} & 85\% \\ 
    \hline
     Total & 152 & 243 & 62\% & 137 & 172 & \blue{80\%} \\ 
  \end{tabular}
\end{center}
\pause
\blue{Hypothetical data}: same Confederate/Union proportions between Dems/Repubs
\begin{center}
	\begin{tabular}{c|ccc|ccc}
     & \multicolumn{3}{c|}{Democrats}  & \multicolumn{3}{c}{Republicans} \\ 
     & Pro & Total & \% & Pro & Total & \% \\ 
     \hline
     Confederate & 8 & 91 & \blue{8\%} & 0 & \textcolor{red}{64} & 0\% \\ 
	 Union & 144 & 152 & \blue{95\%} & 92 & \textcolor{red}{108} & 85\% \\ 
    \hline
     Total & 152 & 243 & \blue{62\%} & 92 & 172 & 53\% \\      
  \end{tabular}
\end{center}

\end{frame}
%------------------------------------------------------------------------------














\end{document}










